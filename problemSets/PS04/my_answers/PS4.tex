\documentclass[12pt,letterpaper]{article}
\usepackage{graphicx,textcomp}
\usepackage{natbib}
\usepackage{setspace}
\usepackage{fullpage}
\usepackage{color}
\usepackage[reqno]{amsmath}
\usepackage{amsthm}
\usepackage{fancyvrb}
\usepackage{amssymb,enumerate}
\usepackage[all]{xy}
\usepackage{endnotes}
\usepackage{lscape}
\newtheorem{com}{Comment}
\usepackage{float}
\usepackage{hyperref}
\newtheorem{lem} {Lemma}
\newtheorem{prop}{Proposition}
\newtheorem{thm}{Theorem}
\newtheorem{defn}{Definition}
\newtheorem{cor}{Corollary}
\newtheorem{obs}{Observation}
\usepackage[compact]{titlesec}
\usepackage{dcolumn}
\usepackage{tikz}
\usetikzlibrary{arrows}
\usepackage{multirow}
\usepackage{xcolor}
\newcolumntype{.}{D{.}{.}{-1}}
\newcolumntype{d}[1]{D{.}{.}{#1}}
\definecolor{light-gray}{gray}{0.65}
\usepackage{url}
\usepackage{listings}
\usepackage{color}

\definecolor{codegreen}{rgb}{0,0.6,0}
\definecolor{codegray}{rgb}{0.5,0.5,0.5}
\definecolor{codepurple}{rgb}{0.58,0,0.82}
\definecolor{backcolour}{rgb}{0.95,0.95,0.92}

\lstdefinestyle{mystyle}{
	backgroundcolor=\color{backcolour},   
	commentstyle=\color{codegreen},
	keywordstyle=\color{magenta},
	numberstyle=\tiny\color{codegray},
	stringstyle=\color{codepurple},
	basicstyle=\footnotesize,
	breakatwhitespace=false,         
	breaklines=true,                 
	captionpos=b,                    
	keepspaces=true,                 
	numbers=left,                    
	numbersep=5pt,                  
	showspaces=false,                
	showstringspaces=false,
	showtabs=false,                  
	tabsize=2
}
\lstset{style=mystyle}
\newcommand{\Sref}[1]{Section~\ref{#1}}
\newtheorem{hyp}{Hypothesis}


\title{Problem Set 4}
\date{Due: December 3, 2023}
\author{Applied Stats/Quant Methods 1}


\begin{document}
	\maketitle
	\section*{Instructions}
	\begin{itemize}
		\item Please show your work! You may lose points by simply writing in the answer. If the problem requires you to execute commands in \texttt{R}, please include the code you used to get your answers. Please also include the \texttt{.R} file that contains your code. If you are not sure if work needs to be shown for a particular problem, please ask.
		\item Your homework should be submitted electronically on GitHub.
		\item This problem set is due before 23:59 on Sunday December 3, 2023. No late assignments will be accepted.
	\end{itemize}



	\vspace{.5cm}
\section*{Question 1: Economics}
\vspace{.25cm}
\noindent 	
In this question, use the \texttt{prestige} dataset in the \texttt{car} library. First, run the following commands:

\begin{verbatim}
install.packages(car)
library(car)
data(Prestige)
help(Prestige)
\end{verbatim} 


\noindent We would like to study whether individuals with higher levels of income have more prestigious jobs. Moreover, we would like to study whether professionals have more prestigious jobs than blue and white collar workers.

\newpage
\begin{enumerate}
	
	\item [(a)]
	Create a new variable \texttt{professional} by recoding the variable \texttt{type} so that professionals are coded as $1$, and blue and white collar workers are coded as $0$ (Hint: \texttt{ifelse}).
	
	\vspace{0.05cm}
		
		\begin{lstlisting}[language=R]
			Prestige$professional <- ifelse(Prestige$type == "professional", 1, 0)
		\end{lstlisting}
		
		This code creates a new variable \texttt{professional} in the \texttt{Prestige} dataset by recoding the variable \texttt{type}. Professionals are coded as 1, and blue and white collar workers are coded as 0.
	\end{enumerate}
	\begin{enumerate}
	\item [(b)]
	Run a linear model with \texttt{prestige} as an outcome and \texttt{income}, \texttt{professional}, and the interaction of the two as predictors (Note: this is a continuous $\times$ dummy interaction.)
	
	\vspace{0.05cm}
		
		\begin{lstlisting}[language=R]
			model <- lm(prestige ~ income * professional, data = Prestige)
		\end{lstlisting}
		
		This code runs a linear model with \texttt{prestige} as the outcome variable and \texttt{income}, \texttt{professional}, and their interaction as predictors.
		
	\end{enumerate}
	
		\begin{enumerate}
	\item [(c)]
	Write the prediction equation based on the result.
	\vspace{0.05cm}
	
	\begin{enumerate}[(c)]
		To write the prediction equation based on the result, you can use the following code in R:
		
		\begin{lstlisting}[language=R]
			summary(model)
		\end{lstlisting}
		
		This code provides a summary of the linear model, including the coefficients and the prediction equation.
						
		\begin{verbatim}
			Call:
			lm(formula = prestige ~ income * professional, data = Prestige)
			
			Residuals:
			Min       1Q   					Median       3Q      Max  
			-32.085 -8.648 -2.498  8.979  31.879 
			
				Coefficients: (2 not defined because of singularities)
			Estimate Std. Error t value Pr(>|t|)    
			(Intercept)         2.760e+01  2.380e+00  11.594  < 2e-16 ***
			income              2.844e-03  2.933e-04   9.694 6.77e-16 ***
			professional               NA         NA      NA       NA    
			income:professional        NA         NA      NA       NA    
			---Signif. codes:  0 ‘***’ 0.001 ‘**’ 0.01 ‘*’ 0.05 ‘.’ 0.1 ‘ ’ 1
			
			Residual standard error: 12.22 on 96 degrees of freedom  
			(4 observations deleted due to missingness)
			Multiple R-squared:  0.4946,	Adjusted R-squared:  0.4894 
			F-statistic: 93.97 on 1 and 96 DF,  p-value: 6.773e-16
		\end{verbatim}
		
	\end{enumerate}

	\item [(d)]
	Interpret the coefficient for \texttt{income}.
	
	\vspace{0.05cm}	
			\begin{lstlisting}[language=R]
		coefficients(summary(model))["income"]
	\end{lstlisting}
	
	[1] NA
	
	\item [(e)]
	Interpret the coefficient for \texttt{professional}.
	
		\vspace{0.05cm}	
	\begin{lstlisting}[language=R]
		coefficients(summary(model))["professional"]
	\end{lstlisting}
	
	[1] NA
	
	\item [(f)]
	What is the effect of a \$1,000 increase in income on prestige score for professional occupations? In other words, we are interested in the marginal effect of income when the variable \texttt{professional} takes the value of $1$. Calculate the change in $\hat{y}$ associated with a \$1,000 increase in income based on your answer for (c).
	
	\vspace{0.05cm}
		\begin{lstlisting}[language=R]
		fitted_value <- fitted(model)new_income <- 1000new_prestige <- predict(model, newdata = data.frame(income = new_income, professional = 1))effect <- new_prestige - fitted_valueeffect
	\end{lstlisting}
	
		\begin{verbatim}
			       gov.administrators          general.managers               accountants
			       -32.277412                -70.745286                -23.519203       
			       purchasing.officers                  chemists                physicists                
			       -22.364712                -21.050981                -28.521050                
			       biologists                architects           civil.engineers                
			       -20.638662                -37.429969                -29.507771          
			       mining.engineers                 surveyors               draughtsmen                
			       -28.501145                -13.939201                -17.229217       
			       computer.programers                economists             psychologists                -21.113539                -20.044355                -18.213094            
			       social.workers                   lawyers                librarians                
			       -15.173313                -51.932198                -14.536352    
			       vocational.counsellors                 ministers       university.teachers                -24.434834                -10.481415                -32.644233   
			       primary.school.teachers secondary.school.teachers                physicians                -13.216933                -20.001702                -69.121605             
			       veterinarians  osteopaths.chiropractors                    nurses                
			       -38.553181                -46.913289                -10.276678             
			       nursing.aides          physio.therapsts               pharmacists                 
			       -7.066282                -11.635906                -26.820593       
			       medical.technicians        commercial.artists       radio.tv.announcers                -11.886141                -14.778056                -18.659535               
			       secretaries                   typists               bookkeepers                 
			       -8.633092                 -6.107998                 -9.520287          
			       tellers.cashiers        computer.operators           shipping.clerks                 
			       -4.117496                 -9.469103                -10.694683               
			       file.clerks              receptionsts             mail.carriers                 
			       -5.732646                 -5.405635                -12.827364             
			       postal.clerks       telephone.operators                collectors                 
			       -7.788550                 -6.144964                -10.637812           
			       claim.adjustors             travel.clerks             office.clerks                
			       -11.522163                -14.954357                 -8.743991         
			       sales.supervisors     commercial.travellers              sales.clerks                
			       -18.432049                -22.123008                 -4.532657 
			       service.station.attendant          insurance.agents      real.estate.salesmen                 -3.895697                -20.277529                -17.038697                    
			       buyers              firefighters                 policemen                
			       -19.779903                -22.450019                -22.438645                     
			       cooks                bartenders         funeral.directors                 
			       -6.017003                 -8.331673                -19.532512                
			       launderers                  janitors        elevator.operators                 
			       -5.687149                 -7.029316                 -7.342109              
			       farm.workers      rotary.well.drillers                    bakers                 
			       -1.865385                -16.663346                 -9.096594            
			       slaughterers.1            slaughterers.2                   canners                
			       -11.755336                -11.755336                 -2.530781           
			       textile.weavers         textile.labourers           tool.die.makers                 
			       -9.790426                 -7.066282                -20.027294                
			       machinists       sheet.metal.workers                   welders                
			       -16.168564                -15.824491                -15.574257              
			       auto.workers          aircraft.workers        electronic.workers                
			       -13.680436                -15.847240                 -8.365796        
			       radio.tv.repairmen     sewing.mach.operators            auto.repairmen                -12.651062                 -5.252082                -13.634939        
			       aircraft.repairmen        railway.sectionmen        electrical.linemen                -19.097445                -10.509851                -20.803590              
			       electricians      construction.foremen                carpenters                
			       -17.479451                -22.407366                -12.224526                    
			       masons            house.painters                  plumbers                
			       -14.101285                -10.091845                -16.856709   
			       construction.labourers                    pilots           train.engineers                 
			       -8.274801                -37.057461                -22.307841               
			       bus.drivers              taxi.drivers              longshoremen                
			       -12.972386                 -9.167684                -10.671934               
			       typesetters               bookbinders                
			       -15.531603                 -7.441634 
		\end{verbatim}
	
	\item [(g)]
	What is the effect of changing one's occupations from non-professional to professional when her income is \$6,000? We are interested in the marginal effect of professional jobs when the variable \texttt{income} takes the value of $6,000$. Calculate the change in $\hat{y}$ based on your answer for (c).
			\vspace{0.05cm}	
	\begin{lstlisting}[language=R]
		new_income <- 6000new_prestige <- predict(model, newdata = data.frame(income = new_income, professional = 1))old_prestige <- predict(model, newdata = data.frame(income = new_income, professional = 0))effect <- new_prestige - old_prestigeeffect
	\end{lstlisting}
	
\end{enumerate}

\section*{Question 2: Political Science}
\vspace{.25cm}
\noindent 	Researchers are interested in learning the effect of all of those yard signs on voting preferences.\footnote{Donald P. Green, Jonathan	S. Krasno, Alexander Coppock, Benjamin D. Farrer,	Brandon Lenoir, Joshua N. Zingher. 2016. ``The effects of lawn signs on vote outcomes: Results from four randomized field experiments.'' Electoral Studies 41: 143-150. } Working with a campaign in Fairfax County, Virginia, 131 precincts were randomly divided into a treatment and control group. In 30 precincts, signs were posted around the precinct that read, ``For Sale: Terry McAuliffe. Don't Sellout Virgina on November 5.'' \\

Below is the result of a regression with two variables and a constant.  The dependent variable is the proportion of the vote that went to McAuliff's opponent Ken Cuccinelli. The first variable indicates whether a precinct was randomly assigned to have the sign against McAuliffe posted. The second variable indicates
a precinct that was adjacent to a precinct in the treatment group (since people in those precincts might be exposed to the signs).  \\

\vspace{.5cm}
\begin{table}[!htbp]
	\centering 
	\textbf{Impact of lawn signs on vote share}\\
	\begin{tabular}{@{\extracolsep{5pt}}lccc} 
		\\[-1.8ex] 
		\hline \\[-1.8ex]
		Precinct assigned lawn signs  (n=30)  & 0.042\\
		& (0.016) \\
		Precinct adjacent to lawn signs (n=76) & 0.042 \\
		&  (0.013) \\
		Constant  & 0.302\\
		& (0.011)
		\\
		\hline \\
	\end{tabular}\\
	\footnotesize{\textit{Notes:} $R^2$=0.094, N=131}
\end{table}

\vspace{.5cm}
\begin{enumerate}
	\item [(a)] Use the results from a linear regression to determine whether having these yard signs in a precinct affects vote share (e.g., conduct a hypothesis test with $\alpha = .05$).
	
	\newpage		
		\begin{lstlisting}[language=R]
	data <- data.frame(  group = c(rep("treatment", 30), rep("control", 76)),  vote_share = c(0.042, 0.042, rep(NA, 104)),  adjacent = c(rep(NA, 30), rep(1, 76)))
	# Run a linear regression
	model <- lm(vote_share ~ group, data = data)
	# Summary of the regression
	summary(model)
	\end{lstlisting}
	
			\begin{verbatim}
		Call:
		lm(formula = prestige ~ income * professional, data = Prestige)
		
		Residuals:
		Min       1Q   					Median       3Q      Max  
		-32.085 -8.648 -2.498  8.979  31.879 
		
		Coefficients: (2 not defined because of singularities)
		Estimate Std. Error t value Pr(>|t|)    
		(Intercept)         2.760e+01  2.380e+00  11.594  < 2e-16 ***
		income              2.844e-03  2.933e-04   9.694 6.77e-16 ***
		professional               NA         NA      NA       NA    
		income:professional        NA         NA      NA       NA    
		---Signif. codes:  0 ‘***’ 0.001 ‘**’ 0.01 ‘*’ 0.05 ‘.’ 0.1 ‘ ’ 1
		
		Residual standard error: 12.22 on 96 degrees of freedom  
		(4 observations deleted due to missingness)
		Multiple R-squared:  0.4946,	Adjusted R-squared:  0.4894 
		F-statistic: 93.97 on 1 and 96 DF,  p-value: 6.773e-16
	\end{verbatim}
	
	\item [(b)]  Use the results to determine whether being
	next to precincts with these yard signs affects vote
	share (e.g., conduct a hypothesis test with $\alpha = .05$).
	
	\vspace{0.05cm}
			\begin{lstlisting}[language=R]
	data <- data.frame(  group = factor(c(rep("treatment", 30), rep("control", 76))),  vote_share = c(0.042, 0.042, rep(NA, 104)),  adjacent = c(rep(NA, 30), rep(1, 76)))
	# Run a linear regression
	model <- lm(vote_share ~ group + adjacent, data = data)
	# Summary of the regression
	summary(model)
	\end{lstlisting}
	\newpage			
	\item [(c)] Interpret the coefficient for the constant term substantively.
	\vspace{0.05cm}
	
				\begin{lstlisting}[language=R]
	data <- data.frame(  group = factor(c(rep("treatment", 30), rep("control", 76))),  vote_share = c(0.042, 0.042, rep(NA, 104)),  adjacent = c(rep(NA, 30), rep(1, 76)))
	# Run a linear regression
	model <- lm(vote_share ~ group + adjacent, data = data)
	# Summary of the regression
	summary(model)
	\end{lstlisting}
	
	\item [(d)] Evaluate the model fit for this regression.  What does this	tell us about the importance of yard signs versus other factors that are not modeled?
	
		\begin{lstlisting}[language=R]
			# Load the data
			data <- data.frame(
			group = factor(c(rep("treatment", 30), rep("control", 76))),
			vote_share = c(0.042, 0.042, rep(NA, 104)),
			adjacent = c(rep(NA, 30), rep(1, 76))
			)
			
			# Run a linear regression
			model <- lm(vote_share ~ group + adjacent, data = data)
			
			# Evaluate the model fit
			summary(model)
		\end{lstlisting}
		
		In the output of the \texttt{summary(model)} function, we can look at the R-squared value, which measures the proportion of the variance in the dependent variable that is predictable from the independent variables. A higher R-squared value indicates a better fit of the model to the data. Additionally, we can examine the p-values for the coefficients of the independent variables to assess their significance in explaining the variation in the dependent variable. This information will help us understand the importance of yard signs versus other factors that are not modeled in the regression.
	
\end{enumerate}  


\end{document}
